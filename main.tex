\documentclass[11pt,twoside]{article}
\usepackage{geometry}
\geometry{a4paper,margin=1cm,footskip=1em}

\usepackage[table]{xcolor}

\usepackage{pgfpages}
\pgfpagesuselayout{2 on 1}[a4paper,border shrink=0pt,landscape]

\usepackage{fontawesome5}
\usepackage{ragged2e}
\usepackage{parskip}

\usepackage{booktabs,makecell,xltabular}

\usepackage{lmodern} % 使用 Latin Modern 字体,支持斜体
\usepackage[T1]{fontenc}
\usepackage[lf,default]{FiraSans}
\renewcommand{\ttdefault}{FiraSans} % Use FiraSans for monospace text

% 改善大写I的显示清晰度 - 使用有衬线的字体,大写I带上下横线
\usepackage{charter} % 使用Charter字体,大写I有非常清晰的上下横线
\renewcommand{\familydefault}{\rmdefault}

\usepackage{regexpatch}
\usepackage[os=mac]{menukeys}
\renewmenumacro{\keys}[+]{shadowedroundedkeys}
\renewmenumacro{\menu}[>]{angularmenus}
\xpatchcmd*{\SPACE}{2em}{1em}{}{}

\renewcommand{\tabularxcolumn}[1]{m{#1}}
\renewcommand{\arraystretch}{1.4}
\arrayrulecolor{gray!60!white}

\makeatletter
\renewcommand{\maketitle}{{\centering\sffamily{\LARGE\bfseries\@title}\par\vskip\baselineskip{\large\@date}\par}\vskip\baselineskip}
% nifty commands by Paul Gaborit from http://tex.stackexchange.com/a/236891/226
\def\setmenukeyswin{\def\tw@mk@os{win}}
\def\setmenukeysmac{\def\tw@mk@os{mac}}
\makeatother

\usepackage{hyperref}
\urlstyle{same}
% latex command displaced in text
\newcommand{\tex}[1]{\textbackslash\texttt{#1}}

\usepackage{amsmath,amssymb,amsfonts,amsthm}

\newcommand{\inputrow}[4]{%
  \makecell{\keys{#1}} & \makecell{\keys{#2}} & $\csname #3\endcsname$ (\tex{#3} in #4) \\* \midrule
}


\title{Liii STEM Input Method Cheatsheet}
\author{Liii Network}

\begin{document}

\maketitle

% \emph{Some keyboard shortcuts in this list may not be available on non-US keyboards or in vim/emacs modes. For example, some input methods on the Mac may use \keys{alt} (equivalent of {\setmenukeyswin\keys{alt}}) for accented characters input instead.}

\emph{\href{https://liiistem.cn/}{Liii STEM}~(\url{https://liiistem.cn/}) is a WYSIWYG editor that can speed up your mathematical writing by 10x. See \href{https://liiistem.com/docs/guide-equation.html}{Quick formula editing} for more details. This is a pdf version cheatsheet of the keys available in Liii STEM Input Method.}

\emph{Unlike the shorcut hints inside Liii STEM. We distinguish the capital and noncapital letters in this cheatsheet; For example, \keys{J} and \keys{j} are different. We also use \keys{\shift j} to replace \keys{J} where \keys{\shift} represents the \texttt{Shift} key.}

\emph{When no plus sign is shown between different keyboard keys, it means they should be pressed in sequence. Alternatively, a plus sign between them means they should be pressed at the same time. For modifier keys (and their combination such as \keys{ctrl \shift}) \keys{\shift}, \keys{ctrl}, \keys{alt} (Windows) or \keys{option} (Mac), and \keys{cmd} (Mac), the plus sign after them means to hold down the modifier key while pressing the next key. For example \keys{Ctrl + f 1} means to hold down the \keys{Ctrl} key and press the \keys{f} key and then press \keys{1} in sequence.}


\emph{All \keys{tab} keys represents tab varient. For example, to insert $\nabla$, press \keys{\shift v} and press \keys{tab} twice. To insert $\Phi$, press \keys{\shift v} and press \keys{tab} once. In the rest of this tutorial, we do specify the exact number of \keys{tab} we used, i.e., the keyboard expression for both \( \nabla \) and \( \Phi \) is \keys{shift v tab}.}


% \emph{The symbol \LaTeX\$ represents that the shortcut is only available in math mode.}




\bigskip

\begin{xltabular}{\textwidth}{
        >{\setmenukeyswin}c @{\hspace{2em}}
        >{\setmenukeysmac}c @{\hspace{2em}}
        >{\renewcommand\cellalign{cl}\RaggedRight\arraybackslash}X}


    \toprule
    \makecell{\sffamily Windows \faWindows\\\sffamily\textsc{gnu}/Linux \faLinux} & \sffamily Mac \faApple & \multicolumn{1}{c}{\sffamily  Equivalent in \LaTeX \faComment}\\
    \midrule
    \endfirsthead

    \footnotesize \faChevronCircleLeft\ (from previous page)\\[1em]
    \toprule
    \makecell{\sffamily Windows \faWindows\\\sffamily\textsc{gnu}/Linux \faLinux} & \sffamily Mac \faApple & \multicolumn{1}{c}{\sffamily Equivalent in  \LaTeX \faComment}\\
    \midrule
    \endhead

    \\[-1.5em]
    \multicolumn{3}{r}{\footnotesize (continued next page) \faChevronCircleRight}
    \endfoot
    \bottomrule
    \endlastfoot



    \makecell{\textbf{Environmental Shortcuts}} \\
\midrule

%%%%%%%%%%%%%%%%%%%% Non-breaking space
\makecell{%
  \keys{space + tab}} &

\makecell{%
  \keys{space + tab}} &
Non-breaking space ($\backslash$nbsp or $\sim$ )
\\*
\midrule
%%%%%%%%%%%%%%%%%%%%%%%%%%%


%%%%%%%%%%%%%%%%%%%% indent
\makecell{%
  \keys{ctrl  +t}} &

\makecell{%
  $\backslash$} &
\tex{indent}
\\*
\midrule
%%%%%%%%%%%%%%%%%%%%%%%%%%%

%%%%%%%%%%%%%%%%%%%% raggedleft 
\makecell{%
  \keys{ctrl +l}} &

\makecell{%
  $\backslash$} &
\tex{raggedleft}
\\*
\midrule
%%%%%%%%%%%%%%%%%%%%%%%%%%%

%%%%%%%%%%%%%%%%%%%% centering 
\makecell{%
  \keys{ctrl +e}} &

\makecell{%
  $\backslash$} &
\tex{centering }
\\*
\midrule
%%%%%%%%%%%%%%%%%%%%%%%%%%%

%%%%%%%%%%%%%%%%%%%% raggedright
\makecell{%
  \keys{ctrl +r}} &

\makecell{%
  $\backslash$} &
\tex{raggedright}
\\*
\midrule
%%%%%%%%%%%%%%%%%%%%%%%%%%%

%%%%%%%%%%%%%%%%%%%% section
\makecell{%
  \keys{alt +1}} &

\makecell{%
  \keys{option +1}} &
\tex{section}
\\*
\midrule
%%%%%%%%%%%%%%%%%%%%%%%%%%%

%%%%%%%%%%%%%%%%%%%% subsection
\makecell{%
  \keys{alt +2}} &

\makecell{%
  \keys{option +2}} &
\tex{subsection}
\\*
\midrule
%%%%%%%%%%%%%%%%%%%%%%%%%%%

%%%%%%%%%%%%%%%%%%%% subsubsection
\makecell{%
  \keys{alt +3}} &

\makecell{%
  \keys{option +3}} &
\tex{subsubsection}
\\*
\midrule
%%%%%%%%%%%%%%%%%%%%%%%%%%%

%%%%%%%%%%%%%%%%%%%% paragraph
\makecell{%
  \keys{alt + 4}} &

\makecell{%
  \keys{option +  4}} &
\tex{paragraph}
\\*
\midrule
%%%%%%%%%%%%%%%%%%%%%%%%%%%

%%%%%%%%%%%%%%%%%%%% subparagraph
\makecell{%
  \keys{alt + 5}} &

\makecell{%
  \keys{option + 5}} &
\tex{subparagraph}
\\*
\midrule
%%%%%%%%%%%%%%%%%%%%%%%%%%%

%%%%%%%%%%%%%%%%%%%% appendix
\makecell{%
  \keys{alt + 6}} &

\makecell{%
  \keys{option + 6}} &
\tex{appendix}
\\*
\midrule

%%%%%%%%%%%%%%%%%%%% itemize
\makecell{%
  \keys{{+} + tab}} &

\makecell{%
  \keys{{+} + tab}} &
\tex{itemize}
\\*
\midrule
%%%%%%%%%%%%%%%%%%%%%%%%%%%

%%%%%%%%%%%%%%%%%%%% enumerate
\makecell{%
\keys{1 + {.} + tab}} &

\makecell{%
\keys{1 + {.} + tab}} &
\tex{enumerate}
\\*
\midrule
%%%%%%%%%%%%%%%%%%%%%%%%%%%

%%%%%%%%%%%%%%%%%%%% inline math mode
\makecell{%
  \keys{\$}} &

\makecell{%
  \keys{\$}} &
inline math mode
\\*
\midrule
%%%%%%%%%%%%%%%%%%%%%%%%%%%

%%%%%%%%%%%%%%%%%%%% single-line math mode
\makecell{%
  \keys{alt + \$}} &

\makecell{%
  \keys{option + \$}} &
single-line math mode
\\*
\midrule
%%%%%%%%%%%%%%%%%%%%%%%%%%%

%%%%%%%%%%%%%%%%%%%% multi-line math mode
\makecell{%
  \keys{alt  + \&}} &

\makecell{%
  \keys{option + \&}} &
multi-line math: eqnarray
\\*
\midrule
%%%%%%%%%%%%%%%%%%%%%%%%%%%

%%%%%%%%%%%%%%%%%%%% multi-line math mode
\makecell{%
  \keys{ctrl + \$}} &

\makecell{%
  \keys{ctrl + \$}} &
multi-line math: align
\\*
\midrule
%%%%%%%%%%%%%%%%%%%%%%%%%%%

%%%%%%%%%%%%%%%%%%%% add equation number
\makecell{%
  \keys{ctrl + \#}} &

\makecell{%
  \keys{ctrl + \#}} &
add equation number
\\*
\midrule
%%%%%%%%%%%%%%%%%%%%%%%%%%%

%%%%%%%%%%%%%%%%%%%% matrix 
\makecell{%
  \keys{alt  + arrow}} &

\makecell{%
  \keys{option + arrow}} &
add new row/column in \textit{matrix/table/choice/stack}
\\*
\midrule
%%%%%%%%%%%%%%%%%%%%%%%%%%%

%%%%%%%%%%%%%%%%%%%% footnote
\makecell{%
  \keys{ctrl \shift + f}} &

\makecell{%
  \keys{ctrl \shift + f}} &
add footnote
\\*
\midrule
%%%%%%%%%%%%%%%%%%%%%%%%%%%

%%%%%%%%%%%%%%%%%%%% add new script
\makecell{%
  \keys{ctrl + n}} &

\makecell{%
  \keys{cmd + n}} &
add new script
\\*
\midrule
%%%%%%%%%%%%%%%%%%%%%%%%%%%

%%%%%%%%%%%%%%%%%%%% export to PDF
\makecell{%
  \keys{ctrl + p}} &

\makecell{%
  \keys{cmd + p}} &
export to PDF
\\*
\midrule
%%%%%%%%%%%%%%%%%%%%%%%%%%%
    \makecell{\textbf{Common Constructs }} \\
\midrule

%%%%%%%%%%%%%%%%%%%% Superscripts
\makecell{%
\keys{x + \^{} + 2}} & 

\makecell{%
\keys{x + \^{} + 2}} & 
$x^2$ ({x\^{}2})
\\*
\midrule
%%%%%%%%%%%%%%%%%%%%%%%%%%%

%%%%%%%%%%%%%%%%%%%% Subscripts
\makecell{%
\keys{x + \_ + \{i,j\}}} & 

\makecell{%
\keys{x + \_ + \{i,j\}}} & 
$x_{i,j}$ ({x\_\{i.j\}})
\\*
\midrule
%%%%%%%%%%%%%%%%%%%%%%%%%%%

%%%%%%%%%%%%%%%%%%%% sqrt
\makecell{%
\keys{Alt + s + 2}} & 

\makecell{%
\keys{option + s + 2}} & 
$\sqrt{2}$ (\tex{sqrt}\{2\})
\\*
\midrule
%%%%%%%%%%%%%%%%%%%%%%%%%%%

%%%%%%%%%%%%%%%%%%%% sqrt
\makecell{%
 \keys{Alt + s  + Tab + 3 + $\leftarrow$ + $\leftarrow$ + n}} & 

\makecell{%
\keys{option + s  + Tab + 3 + $\leftarrow$ + $\leftarrow$ + n}} & 
$\sqrt[n]{3}$ (\tex{sqrt}[n]\{3\})
\\*
\midrule
%%%%%%%%%%%%%%%%%%%%%%%%%%%

%%%%%%%%%%%%%%%%%%%% fraction
\makecell{%
\keys{Alt + f}} & 

\makecell{%
\keys{option + f}} & 
$\frac{2}{3}$ (\tex{frac}\{2\}\{3\})
\\*
\midrule
%%%%%%%%%%%%%%%%%%%%%%%%%%%
    \makecell{\textbf{Font}} \\
\midrule


%%%%%%%%%%%%%%%%%%%% 
\makecell{%
    \keys{A + A}} &

\makecell{%
    \keys{A + A}} &
Blackground $\mathbb{A}$ (\tex{mathbb\{A\}})
\\*
\midrule
%%%%%%%%%%%%%%%%%%%%%%%%%%%

%%%%%%%%%%%%%%%%%%%% Calligraphic letters
\makecell{%
    \keys{F7 + A} or \keys{A + A + tab}}  &

\makecell{%
    \keys{F7 + A} or \keys{A + A + tab}} &
Calligraphic $\mathcal{A}$ (\tex{mathcal\{A\} })
\\*
\midrule
%%%%%%%%%%%%%%%%%%%%%%%%%%%

%%%%%%%%%%%%%%%%%%%% Gothic letters
\makecell{%
    \keys{F8 + A} or \keys{A + A + tab * 2}} &

\makecell{%
    \keys{F8 + A} or \keys{A + A + tab * 2}} &
Gothic $\mathfrak{A}$ (\tex{mathfrak\{A\}})
\\*
\midrule
%%%%%%%%%%%%%%%%%%%%%%%%%%%

%%%%%%%%%%%%%%%%%%%% Bold letters
\makecell{%
    \keys{ctrl + b  + A} or \keys{A + A + shift + tab} } &

\makecell{%
    \keys{cmd + b + A} or \keys{A + A + shift + tab}} &
Bold $\mathbf{A}$ (\tex{mathbf\{A\}})
\\*
\midrule
%%%%%%%%%%%%%%%%%%%%%%%%%%%

%%%%%%%%%%%%%%%%%%%% Italic letters
\makecell{%
    \keys{ctrl + i + A}} &

\makecell{%
    \keys{cmd + i + A}} &
Italic $\mathit{A}$ (\tex{mathit\{A\}})
\\*
\midrule
%%%%%%%%%%%%%%%%%%%%%%%%%%%

    \makecell{\textbf{Greek Letters}} \\
\midrule

%%%%%%%%%%%%%%%%%%%%%%%%%%% alpha
\makecell{%
\keys{a + tab}} & 

\makecell{%
\keys{a+ tab}} & 
$\alpha$ (\tex{alpha})
\\*
\midrule
%%%%%%%%%%%%%%%%%%%%%%%%%%%

%%%%%% beta
\makecell{%
\keys{b + tab}} & 

\makecell{%
\keys{b + tab}} & 
$\beta$ (\tex{beta})
\\*
\midrule
%%%%%%%%%%%%%%%%%%%%%%%%%%%

%%%%%%%%%%%%%%%%%%%%%%%%%% gamma 
\makecell{%
\keys{g + tab}, \keys{G + tab}} & 

\makecell{%
\keys{g + tab}, \keys{G + tab}} & 
$\gamma$ (\tex{gamma}), $\Gamma$ (\tex{Gamma})
\\*
\midrule
%%%%%%%%%%%%%%%%%%%%%%%%%%%

%%%%%% delta 
\makecell{%
\keys{d + tab}, \keys{D + tab}} & 

\makecell{%
\keys{d + tab}, \keys{D + tab}} & 
$\delta$ (\tex{delta}), $\Delta$ (\tex{Delta})
\\*
\midrule
%%%%%%%%%%%%%%%%%%%%%%%%%%%

%%%%%%%%%%%%%%%%%%%%%%%%%%% epsilon 
\makecell{%
\keys{e + tab+ tab+ tab}} & 

\makecell{%
\keys{e+ tab+ tab+ tab}} & 
$\epsilon$ (\tex{epsilon})
\\*
\midrule
%%%%%%%%%%%%%%%%%%%%%%%%%%%

%%%%%%%%%%%%%%%%%%%%%%%%%%% varepsilon 
\makecell{%
\keys{e + tab}} & 

\makecell{%
\keys{e+ tab}} & 
$\varepsilon$ (\tex{varepsilon})
\\*
\midrule
%%%%%%%%%%%%%%%%%%%%%%%%%%%

%%%%%%%%%%%%%%%%%%%%%%%%%%% zeta 
\makecell{%
\keys{z + tab}} & 

\makecell{%
\keys{z+ tab}} & 
$\zeta$ (\tex{zeta})
\\*
\midrule
%%%%%%%%%%%%%%%%%%%%%%%%%%%

%%%%%%%%%%%%%%%%%%%%%%%%%%% eta 
\makecell{%
\keys{h + tab}} & 

\makecell{%
\keys{h+ tab}} & 
$\eta$ (\tex{eta})
\\*
\midrule
%%%%%%%%%%%%%%%%%%%%%%%%%%%

%%%%%%%%%%%%%%%%%%%% theta
\makecell{%
\keys{j + tab}, \keys{J + tab}} & 

\makecell{%
\keys{j + tab}, \keys{J + tab}} & 
$\theta$ (\tex{theta}), $\Theta$ (\tex{Theta})
\\*
\midrule
%%%%%%%%%%%%%%%%%%%%%%%%%%%

%%%%%%%%%%%%%%%%%%%%%%%%%%% vartheta 
\makecell{%
\keys{j + tab+ tab+ tab}} & 

\makecell{%
\keys{j+ tab+ tab+ tab}} & 
$\vartheta$ (\tex{vartheta})
\\*
\midrule
%%%%%%%%%%%%%%%%%%%%%%%%%%%

%%%%%% iota
\makecell{%
\keys{i + tab}} & 

\makecell{%
\keys{i + tab}} & 
$\iota$ (\tex{iota})
\\*
\midrule
%%%%%%%%%%%%%%%%%%%%%%%%%%%

%%%%%% kappa
\makecell{%
\keys{k + tab}} & 

\makecell{%
\keys{k + tab}} & 
$\kappa$ (\tex{kappa})
\\*
\midrule
%%%%%%%%%%%%%%%%%%%%%%%%%%%

%%%%%% lambda
\makecell{%
\keys{l + tab}, \keys{L+ tab}} & 

\makecell{%
\keys{l + tab}, \keys{L+ tab}} & 
$\lambda$ (\tex{lambda}), $\Lambda$ (\tex{Lambda})
\\*
\midrule
%%%%%%%%%%%%%%%%%%%%%%%%%%%

%%%%%%%%%%%%%%%%%%%% mu
\makecell{%
\keys{m + tab}} & 

\makecell{%
\keys{m + tab}} & 
$\mu$ (\tex{mu})
\\*
\midrule
%%%%%%%%%%%%%%%%%%%%%%%%%%%

%%%%%%%%%%%%%%%%%%%%%%%%%%% nu 
\makecell{%
\keys{n + tab}} & 

\makecell{%
\keys{n+ tab}} & 
$\nu$ (\tex{nu})
\\*
\midrule
%%%%%%%%%%%%%%%%%%%%%%%%%%%

%%%%%%%%%%%%%%%%%%%%%%%%%%% xi 
\makecell{%
\keys{x + tab}, \keys{X+ tab}} & 

\makecell{%
\keys{x+ tab}, \keys{X+ tab}} & 
$\xi$ (\tex{xi}), $\Xi$ (\tex{Xi})
\\*
\midrule
%%%%%%%%%%%%%%%%%%%%%%%%%%%

%%%%%%%%%%%%%%%%%%%% pi
\makecell{%
\keys{p + tab}, \keys{P + tab}} & 

\makecell{%
\keys{p + tab}, \keys{P + tab}} & 
$\pi$ (\tex{pi}), $\Pi$ (\tex{Pi})
\\*
\midrule
%%%%%%%%%%%%%%%%%%%%%%%%%%%

% %%%%%%%%%%%%%%%%%%%%%%%%%%% omicron  
% \makecell{%
% \keys{o + tab}} & 

% \makecell{%
% \keys{o+ tab}} & 
% $\omicron$ (\tex{omicron } in \LaTeX\$)
% \\*
% \midrule
% %%%%%%%%%%%%%%%%%%%%%%%%%%%

%%%%%%%%%%%%%%%%%%%%%%%%%%% varpi  
\makecell{%
\keys{p + tab+ tab+ tab}} & 

\makecell{%
\keys{p+ tab+ tab+ tab}} & 
$\varpi$ (\tex{varpi })
\\*
\midrule
%%%%%%%%%%%%%%%%%%%%%%%%%%%

%%%%%%%%%%%%%%%%%%%%%%%%%%% rho 
\makecell{%
\keys{r+ tab}} & 

\makecell{%
\keys{r+ tab}} & 
$\rho$ (\tex{rho})
\\*
\midrule
%%%%%%%%%%%%%%%%%%%%%%%%%%%

%%%%%%%%%%%%%%%%%%%%%%%%%%% varrho 
\makecell{%
\keys{r+ tab+ tab}} & 

\makecell{%
\keys{r+ tab+ tab}} & 
$\varrho$ (\tex{varrho})
\\*
\midrule
%%%%%%%%%%%%%%%%%%%%%%%%%%%

%%%%%%%%%%%%%%%%%%%%%%%%%%% sigma 
\makecell{%
\keys{s+ tab}, \keys{S+ tab}} & 

\makecell{%
\keys{s+ tab}, \keys{S+ tab}} & 
$\sigma$ (\tex{sigma}), $\Sigma$ (\tex{Sigma})
\\*
\midrule
%%%%%%%%%%%%%%%%%%%%%%%%%%%

%%%%%%%%%%%%%%%%%%%%%%%%%%% varsigma
\makecell{%
\keys{s+ tab+ tab}} & 

\makecell{%
\keys{s+ tab+ tab}} & 
$\varsigma$ (\tex{varsigma})
\\*
\midrule
%%%%%%%%%%%%%%%%%%%%%%%%%%%

%%%%%%%%%%%%%%%%%%%%%%%%%%% tau 
\makecell{%
\keys{t+ tab}} & 

\makecell{%
\keys{t+ tab}} & 
$\tau$ (\tex{tau})
\\*
\midrule
%%%%%%%%%%%%%%%%%%%%%%%%%%%

%%%%%%%%%%%%%%%%%%%%%%%%%%% upsilon  
\makecell{%
\keys{u+ tab}, \keys{U+ tab}} & 

\makecell{%
\keys{u+ tab}, \keys{U+ tab}} & 
$\upsilon$ (\tex{upsilon}), $\Upsilon$ (\tex{Upsilon})
\\*
\midrule
%%%%%%%%%%%%%%%%%%%%%%%%%%%

%%%%%%%%%%%%%%%%%%%%%%%%%%% phi  
\makecell{%
\keys{f+ tab+ tab}, \keys{F+ tab+ tab}} & 

\makecell{%
\keys{f+ tab+ tab}, \keys{F tab+ tab}} & 
$\phi$ (\tex{phi}), $\Phi$ (\tex{Phi})
\\*
\midrule
%%%%%%%%%%%%%%%%%%%%%%%%%%%

%%%%%%%%%%%%%%%%%%%%%%%%%%% varphi  
\makecell{%
\keys{f+ tab}} & 

\makecell{%
\keys{f+ tab}} & 
$\varphi$ (\tex{varphi})
\\*
\midrule
%%%%%%%%%%%%%%%%%%%%%%%%%%%

%%%%%%%%%%%%%%%%%%%%%%%%%%% chi 
\makecell{%
\keys{q+ tab}} & 

\makecell{%
\keys{q+ tab}} & 
$\chi$ (\$\tex{chi }\$)
\\*
\midrule
%%%%%%%%%%%%%%%%%%%%%%%%%%%

%%%%%%%%%%%%%%%%%%%%%%%%%%% psi
\makecell{%
\keys{y+ tab}, \keys{Y+ tab}} & 

\makecell{%
\keys{y+ tab}, \keys{Y+ tab}} & 
$\psi$ (\tex{psi}), $\Psi$ (\tex{Psi})
\\*
\midrule
%%%%%%%%%%%%%%%%%%%%%%%%%%%

%%%%%%%%%%%%%%%%%%%%%%%%%%% omega
\makecell{%
\keys{w+ tab},\keys{W+ tab}} & 

\makecell{%
\keys{w+ tab}, \keys{W+ tab}} & 
$\omega$ (\tex{omega}), $\Omega$ (\tex{Omega})
\\*
\midrule
%%%%%%%%%%%%%%%%%%%%%%%%%%%

    \makecell{\textbf{Sets and Logic}} \\
\midrule

%%%%%%%%%%%%%%%%%%%%%%%%%%% cup 
\makecell{%
\keys{\%{} + tab}} & 

\makecell{%
\keys{\%{} + tab}} & 
$\cup$ (\tex{cup})
\\*
\midrule
%%%%%%%%%%%%%%%%%%%%%%%%%%%

%%%%%%%%%%%%%%%%%%%%%%%%%%% cap 
\makecell{%
\keys{\&{} + tab}} & 

\makecell{%
\keys{\&{} + tab}} & 
$\cap$ (\tex{cap})
\\*
\midrule
%%%%%%%%%%%%%%%%%%%%%%%%%%%

%%%%%%%%%%%%%%%%%%%% subset
\makecell{%
\keys{< + tab + tab}} & 

\makecell{%
\keys{< + tab + tab}} & 
$\subset$ (\tex{subset})
\\*
\midrule
%%%%%%%%%%%%%%%%%%%%%%%%%%%

%%%%%%%%%%%%%%%%%%%% subseteq
\makecell{%
\keys{< + tab + tab + =}} & 

\makecell{%
\keys{< + tab + tab + =}} & 
$\subseteq$ (\tex{subseteq})
\\*
\midrule
%%%%%%%%%%%%%%%%%%%%%%%%%%%

%%%%%%%%%%%%%%%%%%%% supset
\makecell{%
\keys{> + tab + tab}} & 

\makecell{%
\keys{> + tab + tab}} & 
$\supset$ (\tex{supset})
\\*
\midrule
%%%%%%%%%%%%%%%%%%%%%%%%%%%

%%%%%%%%%%%%%%%%%%%% supseteq
\makecell{%
\keys{> + tab + tab + =}} & 

\makecell{%
\keys{> + tab + tab + =}} & 
$\supseteq$ (\tex{supseteq})
\\*
\midrule
%%%%%%%%%%%%%%%%%%%%%%%%%%%


%%%%%%%%%%%%%%%%%%%% in
\makecell{%
\keys{< + tab}} & 

\makecell{%
\keys{< + tab}} & 
$\in$ (\tex{in})
\\*
\midrule
%%%%%%%%%%%%%%%%%%%%%%%%%%%

%%%%%%%%%%%%%%%%%%%% ni
\makecell{%
\keys{> + tab}} & 

\makecell{%
\keys{> + tab}} & 
$\ni$ (\tex{ni})
\\*
\midrule
%%%%%%%%%%%%%%%%%%%%%%%%%%%

%%%%%%%%%%%%%%%%%%%% notin
\makecell{%
\keys{< + tab + /}} & 

\makecell{%
\keys{< + tab + /}} & 
$\notin$ (\tex{notin})
\\*
\midrule
%%%%%%%%%%%%%%%%%%%%%%%%%%%

%%%%%%%%%%%%%%%%%%%% mathbb{R}
\makecell{%
\keys{R + R}} & 

\makecell{%
\keys{R + R}} & 
$\mathbb{R}$ (\tex{mathbb\{R\}})
\\*
\midrule
%%%%%%%%%%%%%%%%%%%%%%%%%%%

%%%%%%%%%%%%%%%%%%%% mathbb{Z}
\makecell{%
\keys{Z + Z}} & 

\makecell{%
\keys{Z + Z}} & 
$\mathbb{Z}$ (\tex{mathbb\{Z\}})
\\*
\midrule
%%%%%%%%%%%%%%%%%%%%%%%%%%%

%%%%%%%%%%%%%%%%%%%% mathbb{Q}
\makecell{%
\keys{Q + Q}} & 

\makecell{%
\keys{Q + Q}} & 
$\mathbb{Q}$ (\tex{mathbb\{Q\}})
\\*
\midrule
%%%%%%%%%%%%%%%%%%%%%%%%%%%

%%%%%%%%%%%%%%%%%%%% mathbb{N}
\makecell{%
\keys{N + N}} & 

\makecell{%
\keys{N + N}} & 
$\mathbb{N}$ (\tex{mathbb\{N\}})
\\*
\midrule
%%%%%%%%%%%%%%%%%%%%%%%%%%%

%%%%%%%%%%%%%%%%%%%% mathbb{C}
\makecell{%
\keys{C + C}} & 

\makecell{%
\keys{C + C}} & 
$\mathbb{C}$ (\tex{mathbb\{C\}})
\\*
\midrule
%%%%%%%%%%%%%%%%%%%%%%%%%%%

%%%%%%%%%%%%%%%%%%%% varnothing
\makecell{%
\keys{@ + /}} & 

\makecell{%
\keys{@ + /}} & 
$\varnothing$ (\tex{varnothing})
\\*
\midrule
%%%%%%%%%%%%%%%%%%%%%%%%%%%

% %%%%%%%%%%%%%%%%%%%% emptyset
% \makecell{%
% \keys{@ + /}} & 

% \makecell{%
% \keys{@ + /}} & 
% $\emptyset$ (\tex{emptyset})
% \\*
% \midrule
% %%%%%%%%%%%%%%%%%%%%%%%%%%%

%%%%%%%%%%%%%%%%%%%% aleph    
\makecell{%
\keys{A + tab + tab + tab}} & 

\makecell{%
\keys{A + tab + tab + tab}} & 
$\aleph$ (\tex{aleph})
\\*
\midrule
%%%%%%%%%%%%%%%%%%%%%%%%%%%

% %%%%%%%%%%%%%%%%%%%% setminus    
% \makecell{%
% \keys{= + tab + tab}} & 

% \makecell{%
% \keys{= + tab + tab}} & 
% $\setminus$ (\tex{setminus})
% \\*
% \midrule
% %%%%%%%%%%%%%%%%%%%%%%%%%%%

%%%%%%%%%%%%%%%%%%%% equiv    
\makecell{%
\keys{= + tab + tab}} & 

\makecell{%
\keys{= + tab + tab}} & 
$\equiv$ (\tex{equiv})
\\*
\midrule
%%%%%%%%%%%%%%%%%%%%%%%%%%%

%%%%%%%%%%%%%%%%%%%% forall       
\makecell{%
\keys{A + tab + tab}} & 

\makecell{%
\keys{A + tab + tab}} & 
$\forall$ (\tex{forall  })
\\*
\midrule
%%%%%%%%%%%%%%%%%%%%%%%%%%%

%%%%%%%%%%%%%%%%%%%% exists        
\makecell{%
\keys{E + tab + tab}} & 

\makecell{%
\keys{E + tab + tab}} & 
$\exists$ (\tex{exists   })
\\*
\midrule
%%%%%%%%%%%%%%%%%%%%%%%%%%%

%%%%%%%%%%%%%%%%%%%% neg        
\makecell{%
\keys{! + tab}} & 

\makecell{%
\keys{! + tab}} & 
$\neg$ (\tex{neg  })
\\*
\midrule
%%%%%%%%%%%%%%%%%%%%%%%%%%%

% %%%%%%%%%%%%%%%%%%%% Box        
% \makecell{%
% \keys{@ + tab}} & 

% \makecell{%
% \keys{@ + tab }} & 
% $\Box$ (\tex{Box })
% \\*
% \midrule
% %%%%%%%%%%%%%%%%%%%%%%%%%%%

%%%%%%%%%%%%%%%%%%%% vee
\makecell} & 

\makecell} & 
$\vee$ (\tex{vee})
\\*
\midrule
%%%%%%%%%%%%%%%%%%%%%%%%%%%

%%%%%%%%%%%%%%%%%%%% wedge
\makecell{%
\keys{\&}} & 

\makecell{%
\keys{\&}} & 
$\wedge$ (\tex{wedge})
\\*
\midrule
%%%%%%%%%%%%%%%%%%%%%%%%%%%

%%%%%%%%%%%%%%%%%%%% vdash        
\makecell{%
\keys{|+ tab + -}} & 

\makecell{%
\keys{|+ tab + -}} & 
$\vdash$ (\tex{vdash })
\\*
\midrule
%%%%%%%%%%%%%%%%%%%%%%%%%%%

% %%%%%%%%%%%%%%%%%%%% Vdash        
% \makecell{%
% \keys{|+ | + tab + -}} & 

% \makecell{%
% \keys{|+ | + tab + -}} & 
% $\Vdash$ (\tex{Vdash })
% \\*
% \midrule
% %%%%%%%%%%%%%%%%%%%%%%%%%%%

% %%%%%%%%%%%%%%%%%%%% Vvdash        
% \makecell{%
% \keys{|+ | + | + tab + -}} & 

% \makecell{%
% \keys{|+ | + | + tab + -}} & 
% $\Vvdash$ (\tex{Vvdash })
% \\*
% \midrule
% %%%%%%%%%%%%%%%%%%%%%%%%%%%

% %%%%%%%%%%%%%%%%%%%% dashv        
% \makecell{%
% \keys{- + | +  tab }} & 

% \makecell{%
% \keys{- + | +  tab }} & 
% $\dashv$ (\tex{dashv })
% \\*
% \midrule
% %%%%%%%%%%%%%%%%%%%%%%%%%%%

%%%%%%%%%%%%%%%%%%%% models        
\makecell{%
\keys{|+  tab + =}} & 

\makecell{%
\keys{|+ tab + =}} & 
$\models$ (\tex{models})
\\*
\midrule
%%%%%%%%%%%%%%%%%%%%%%%%%%%

%%%%%%%%%%%%%%%%%%%% rightarrow
\makecell{%
\keys{= + >}} & 

\makecell{%
\keys{= + >}} & 
$\Rightarrow$ (\tex{Rightarrow})
\\*
\midrule
%%%%%%%%%%%%%%%%%%%%%%%%%%%

% %%%%%%%%%%%%%%%%%%%% to
% \makecell{%
% \keys{- + - + >}} & 

% \makecell{%
% \keys{- + - + >}} & 
% $\longrightarrow$ (\tex{longrightarrow} in \LaTeX\$)
% \\*
% \midrule
% %%%%%%%%%%%%%%%%%%%%%%%%%%%

%%%%%%%%%%%%%%%%%%%% nrightarrow
\makecell{%
\keys{= + > + /}} & 

\makecell{%
\keys{= + > + /}} & 
$\nRightarrow$ (\tex{nRightarrow})
\\*
\midrule
%%%%%%%%%%%%%%%%%%%%%%%%%%%
    \makecell{\textbf{Decorations }} \\
\midrule

%%%%%%%%%%%%%%%%%%%% dot
\makecell{%
\keys{\Alt + \.{}}} & 

\makecell{%
\keys{option + \.{}}} & 
$\dot{}$ (\tex{dot\{\}})
\\*
\midrule
%%%%%%%%%%%%%%%%%%%%%%%%%%%

%%%%%%%%%%%%%%%%%%%% ddot
\makecell{%
\keys{\Alt + \.{} + \.{}}} & 

\makecell{%
\keys{option + \.{} + \.{}}} & 
$\ddot{}$ (\tex{ddot\{\}})
\\*
\midrule
%%%%%%%%%%%%%%%%%%%%%%%%%%%

%%%%%%%%%%%%%%%%%%%% hat
\makecell{%
\keys{\Alt + \^{}}} & 

\makecell{%
\keys{option + \^{}}} & 
$\hat{}$ (\tex{hat\{\}})
\\*
\midrule
%%%%%%%%%%%%%%%%%%%%%%%%%%%

%%%%%%%%%%%%%%%%%%%% tilda
\makecell{%
\keys{\Alt + \~{}}} & 

\makecell{%
\keys{option + \~{}}} & 
$\tilde{}$ (\tex{tilde\{\}})
\\*
\midrule
%%%%%%%%%%%%%%%%%%%%%%%%%%%

%%%%%%%%%%%%%%%%%%%% bar
\makecell{%
\keys{\Alt + -{}}} & 

\makecell{%
\keys{option + -{}}} & 
$\bar{}$ (\tex{bar\{\}})
\\*
\midrule
%%%%%%%%%%%%%%%%%%%%%%%%%%%

% %%%%%%%%%%%%%%%%%%%% vec
% \makecell{%
% \keys{\Alt + \~{}}} & 

% \makecell{%
% \keys{option + \~{}}} & 
% $\vec{}$ (\tex{vec\{\}})
% \\*
% \midrule
% %%%%%%%%%%%%%%%%%%%%%%%%%%%

% %%%%%%%%%%%%%%%%%%%% underset
% \makecell{%
% \keys{\Alt + b}} & 

% \makecell{%
% \keys{option + b}} & 
% $\underset{x \in X}{minimize}$ (\tex{underset} in \LaTeX\$)
% \\*
% \midrule
% %%%%%%%%%%%%%%%%%%%%%%%%%%%



    \makecell{\textbf{Dots}} \\
\midrule

%%%%%%%%%%%%%%%%%%%% ldots            
\makecell{%
    \keys{. + . }} &

\makecell{%
    \keys{. + . }} &
$\ldots$ (\tex{ldots  })
\\*
\midrule
%%%%%%%%%%%%%%%%%%%%%%%%%%%

%%%%%%%%%%%%%%%%%%%% cdots            
\makecell{%
    \keys{. + .  + Tab }} &

\makecell{%
    \keys{. + .  + Tab }} &
$\cdots$ (\tex{cdots   })
\\*
\midrule
%%%%%%%%%%%%%%%%%%%%%%%%%%%

%%%%%%%%%%%%%%%%%%%% hdots            
\makecell{%
    \keys{. + .  + Tab + Tab }} &

\makecell{%
    \keys{. + .  + Tab + Tab }} &
high dots
\\*
\midrule
%%%%%%%%%%%%%%%%%%%%%%%%%%%

%%%%%%%%%%%%%%%%%%%% vdots            
\makecell{%
    \keys{. + .  + Tab + Tab  + Tab }} &

\makecell{%
    \keys{. + .   + Tab + Tab  + Tab }} &
$\vdots$ (\tex{vdots  })
\\*
\midrule
%%%%%%%%%%%%%%%%%%%%%%%%%%%

%%%%%%%%%%%%%%%%%%%% ddots            
\makecell{%
    \keys{. + .   + Tab + Tab + Tab + Tab }} &

\makecell{%
    \keys{. + .  + Tab + Tab + Tab + Tab }} &
$\ddots$ (\tex{ddots  })
\\*
\midrule
%%%%%%%%%%%%%%%%%%%%%%%%%%%

%%%%%%%%%%%%%%%%%%%% bddots            
\makecell{%
    \keys{. + .   + Tab + Tab + Tab + Tab + Tab }} &

\makecell{%
    \keys{. + .  + Tab + Tab + Tab + Tab + Tab  }} &
back-diagonal  dots
\\*
\midrule
%%%%%%%%%%%%%%%%%%%%%%%%%%%
    \makecell{\textbf{Other Symbols}} \\
\midrule

%%%%%%%%%%%%%%%%%%%% leq
\makecell{%
    \keys{< + = + tab}} &

\makecell{%
    \keys{< + = + tab}} &
$\leq$ (\tex{leq})
\\*
\midrule
%%%%%%%%%%%%%%%%%%%%%%%%%%%

%%%%%%%%%%%%%%%%%%%% geq
\makecell{%
    \keys{> + = + tab}} &

\makecell{%
    \keys{> + = + tab}} &
$\geq$ (\tex{geq})
\\*
\midrule
%%%%%%%%%%%%%%%%%%%%%%%%%%%

%%%%%%%%%%%%%%%%%%%% neq
\makecell{%
    \keys{ = + \textbackslash}} &

\makecell{%
    \keys{ = + \textbackslash}}&
$\neq$ (\tex{neq})
\\*
\midrule
%%%%%%%%%%%%%%%%%%%%%%%%%%%

%%%%%%%%%%%%%%%%%%%% ll
\makecell{%
    \keys{ < + <}} &

\makecell{%
    \keys{ < + <}} &
$\ll$ (\tex{ll})
\\*
\midrule
%%%%%%%%%%%%%%%%%%%%%%%%%%%

%%%%%%%%%%%%%%%%%%%% gg
\makecell{%
    \keys{ > + >}} &

\makecell{%
    \keys{ > + >}} &
$\gg$ (\tex{gg})
\\*
\midrule
%%%%%%%%%%%%%%%%%%%%%%%%%%%

%%%%%%%%%%%%%%%%%%%% approx
\makecell{%
\keys{ \~{}+ \~{}}} &

\makecell{%
\keys{ \~{} + \~{}}} &
$\approx$ (\tex{approx})
\\*
\midrule
%%%%%%%%%%%%%%%%%%%%%%%%%%%

%%%%%%%%%%%%%%%%%%%% asymp
\makecell{%
    \keys{= +tab}} &

\makecell{%
    \keys{= +tab}} &
$\asymp$ (\tex{asymp})
\\*
\midrule
%%%%%%%%%%%%%%%%%%%%%%%%%%%

%%%%%%%%%%%%%%%%%%%% prec
\makecell{%
    \keys{< +tab }} &

\makecell{%
    \keys{< +tab }} &
$\prec$ (\tex{prec})
\\*
\midrule
%%%%%%%%%%%%%%%%%%%%%%%%%%%

%%%%%%%%%%%%%%%%%%%% preceq
\makecell{%
    \keys{< +tab  + = + tab}} &

\makecell{%
    \keys{< +tab  + = + tab}} &
$\preceq$ (\tex{preceq})
\\*
\midrule
%%%%%%%%%%%%%%%%%%%%%%%%%%%

%%%%%%%%%%%%%%%%%%%% succ
\makecell{%
    \keys{> +tab }} &

\makecell{%
    \keys{> +tab }} &
$\succ$ (\tex{succ})
\\*
\midrule
%%%%%%%%%%%%%%%%%%%%%%%%%%%

%%%%%%%%%%%%%%%%%%%% succeq
\makecell{%
    \keys{> +tab  + = + tab}} &

\makecell{%
    \keys{> +tab  + = + tab}} &
$\succeq$ (\tex{succeq})
\\*
\midrule
%%%%%%%%%%%%%%%%%%%%%%%%%%%

%%%%%%%%%%%%%%%%%%%% propto
\makecell{%
    \keys{@ +@ + tab }} &

\makecell{%
    \keys{@ +@ + tab }} &
$\propto$ (\tex{propto})
\\*
\midrule
%%%%%%%%%%%%%%%%%%%%%%%%%%%

%%%%%%%%%%%%%%%%%%%% doteq
\makecell{%
    \keys{. + =}} &

\makecell{%
    \keys{. + =}} &
$\doteq$ (\tex{doteq})
\\*
\midrule
%%%%%%%%%%%%%%%%%%%%%%%%%%%

%%%%%%%%%%%%%%%%%%%% angle        
\makecell{%
    \keys{@ + tab }} &

\makecell{%
    \keys{@ + tab }} &
$\angle$ (\tex{angle})
\\*
\midrule
%%%%%%%%%%%%%%%%%%%%%%%%%%%

%%%%%%%%%%%%%%%%%%%% ell    
\makecell{%
    \keys{l + tab }} &

\makecell{%
    \keys{l + tab }} &
$\ell$ (\tex{ell})
\\*
\midrule
%%%%%%%%%%%%%%%%%%%%%%%%%%%

%%%%%%%%%%%%%%%%%%%% parallel    
\makecell{%
    \keys{\shift +  F5 + B}} &

\makecell{%
    \keys{\shift +  F5 + B}} &
$\parallel $ (\tex{parallel})
\\*
\midrule
%%%%%%%%%%%%%%%%%%%%%%%%%%%

%%%%%%%%%%%%%%%%%%%% cong     
\makecell{%
\keys{\~{} + =}} &

\makecell{%
\keys{\~{} + =}} &
$\cong $ (\tex{cong})
\\*
\midrule
%%%%%%%%%%%%%%%%%%%%%%%%%%%

%%%%%%%%%%%%%%%%%%%% ncong     
\makecell{%
\keys{\~{} + = + /}} &

\makecell{%
\keys{\~{} + = + /}} &
$\ncong $ (\tex{ncong})
\\*
\midrule
%%%%%%%%%%%%%%%%%%%%%%%%%%%

%%%%%%%%%%%%%%%%%%%% sim     
\makecell{%
\keys{\~{}}} &

\makecell{%
\keys{\~{}}} &
$\sim $ (\tex{sim})
\\*
\midrule
%%%%%%%%%%%%%%%%%%%%%%%%%%%

%%%%%%%%%%%%%%%%%%%% simeq     
\makecell{%
\keys{\~{} + - }} &

\makecell{%
\keys{\~{} + - }} &
$\simeq $ (\tex{simeq})
\\*
\midrule
%%%%%%%%%%%%%%%%%%%%%%%%%%%

%%%%%%%%%%%%%%%%%%%% nsim     
\makecell{%
\keys{\~{} + / }} &

\makecell{%
\keys{\~{} + / }} &
$\nsim $ (\tex{nsim})
\\*
\midrule
%%%%%%%%%%%%%%%%%%%%%%%%%%%

%%%%%%%%%%%%%%%%%%%% oplus
\makecell{%
\keys{@ + {+}}} &

\makecell{%
\keys{@ + {+}}} &
$\oplus$ (\tex{oplus})
\\*
\midrule
%%%%%%%%%%%%%%%%%%%%%%%%%%%

%%%%%%%%%%%%%%%%%%%% ominus
\makecell{%
\keys{@ + {-}}} &

\makecell{%
\keys{@ + {-}}} &
$\ominus$ (\tex{ominus})
\\*
\midrule
%%%%%%%%%%%%%%%%%%%%%%%%%%%

%%%%%%%%%%%%%%%%%%%% odot
\makecell{%
\keys{@ + {.}}} &

\makecell{%
\keys{@ + {.}}} &
$\odot$ (\tex{odot})
\\*
\midrule
%%%%%%%%%%%%%%%%%%%%%%%%%%%

%%%%%%%%%%%%%%%%%%%% otimes
\makecell{%
    \keys{@ + *}} &

\makecell{%
    \keys{@ + *}} &
$\otimes$ (\tex{otimes})
\\*
\midrule
%%%%%%%%%%%%%%%%%%%%%%%%%%%

%%%%%%%%%%%%%%%%%%%% oslash
\makecell{%
    \keys{@ + /}} &

\makecell{%
    \keys{@ + /}} &
$\oslash$ (\tex{oslash})
\\*
\midrule
%%%%%%%%%%%%%%%%%%%%%%%%%%%

%%%%%%%%%%%%%%%%%%%% upharpoonright 
\makecell{%
    \keys{/ + - + tab }} &

\makecell{%
    \keys{/ + - + tab }} &
$\upharpoonright$ (\tex{upharpoonright })
\\*
\midrule
%%%%%%%%%%%%%%%%%%%%%%%%%%%

%%%%%%%%%%%%%%%%%%%% cdot
\makecell{%
    \keys{. + tab }} &

\makecell{%
    \keys{* + tab }} &
$\cdot$ (\tex{cdot})
\\*
\midrule
%%%%%%%%%%%%%%%%%%%%%%%%%%%

%%%%%%%%%%%%%%%%%%%% pm
\makecell{%
\keys{{+} + {-} }} &

\makecell{%
\keys{{+} + {- }}} &
$\pm$ (\tex{pm})
\\*
\midrule
%%%%%%%%%%%%%%%%%%%%%%%%%%%

%%%%%%%%%%%%%%%%%%%% mp
\makecell{%
\keys{{-} + {+} }} &

\makecell{%
\keys{{-} + {+} }} &
$\mp$ (\tex{mp})
\\*
\midrule
%%%%%%%%%%%%%%%%%%%%%%%%%%%

%%%%%%%%%%%%%%%%%%%% times
\makecell{%
    \keys{* + tab}} &

\makecell{%
    \keys{* + tab}} &
$\times$ (\tex{times})
\\*
\midrule
%%%%%%%%%%%%%%%%%%%%%%%%%%%

%%%%%%%%%%%%%%%%%%%% div
\makecell{%
    \keys{/ + tab }} &

\makecell{%
    \keys{/ + tab }} &
$\div$ (\tex{div})
\\*
\midrule
%%%%%%%%%%%%%%%%%%%%%%%%%%%

%%%%%%%%%%%%%%%%%%%% ast
\makecell{%
    \keys{* + tab }} &

\makecell{%
    \keys{* + tab }} &
$\ast$ (\tex{ast})
\\*
\midrule
%%%%%%%%%%%%%%%%%%%%%%%%%%%

%%%%%%%%%%%%%%%%%%%% partial     
\makecell{%
    \keys{d + tab }} &

\makecell{%
    \keys{d + tab }} &
$\partial$ (\tex{partial })
\\*
\midrule
%%%%%%%%%%%%%%%%%%%%%%%%%%%

%%%%%%%%%%%%%%%%%%%% nabla      
\makecell{%
    \keys{V + tab }} &

\makecell{%
    \keys{V + tab }} &
$\nabla$ (\tex{nabla })
\\*
\midrule
%%%%%%%%%%%%%%%%%%%%%%%%%%%

%%%%%%%%%%%%%%%%%%%% circ      
\makecell{%
    \keys{@}} &

\makecell{%
    \keys{@}} &
$\circ$ (\tex{circ})
\\*
\midrule
%%%%%%%%%%%%%%%%%%%%%%%%%%%

%%%%%%%%%%%%%%%%%%%% star      
\makecell{%
    \keys{* + tab }} &

\makecell{%
    \keys{* + tab }} &
$\star $ (\tex{star })
\\*
\midrule
%%%%%%%%%%%%%%%%%%%%%%%%%%%

%%%%%%%%%%%%%%%%%%%% imath     
\makecell{%
    \keys{i + tab  }} &

\makecell{%
    \keys{i + tab  }} &
$\imath$ (\tex{imath  })
\\*
\midrule
%%%%%%%%%%%%%%%%%%%%%%%%%%%

%%%%%%%%%%%%%%%%%%%% jmath     
\makecell{%
    \keys{j + tab  }} &

\makecell{%
    \keys{j + tab  }} &
$\jmath$ (\tex{jmath  })
\\*
\midrule
%%%%%%%%%%%%%%%%%%%%%%%%%%%

%%%%%%%%%%%%%%%%%%%% hbar     
\makecell{%
    \keys{h  + tab  }} &

\makecell{%
    \keys{h + tab  }} &
$\hbar$ (\tex{hbar  })
\\*
\midrule
%%%%%%%%%%%%%%%%%%%%%%%%%%%

%%%%%%%%%%%%%%%%%%%% beth   
\makecell{%
    \keys{B + tab }} &

\makecell{%
    \keys{B + tab }} &
$\beth$ (\tex{beth })
\\*
\midrule
%%%%%%%%%%%%%%%%%%%%%%%%%%%

%%%%%%%%%%%%%%%%%%%% gimel   
\makecell{%
    \keys{G + tab }} &

\makecell{%
    \keys{G + tab }} &
$\gimel$ (\tex{gimel })
\\*
\midrule
%%%%%%%%%%%%%%%%%%%%%%%%%%%

%%%%%%%%%%%%%%%%%%%% daleth  
\makecell{%
    \keys{D + tab }} &

\makecell{%
    \keys{D + tab }} &
$\daleth$ (\tex{daleth})
\\*
\midrule
%%%%%%%%%%%%%%%%%%%%%%%%%%%

%%%%%%%%%%%%%%%%%%%% Re 
\makecell{%
    \keys{R + E }} &

\makecell{%
    \keys{R + E }} &
$\Re$ (\tex{Re})
\\*
\midrule
%%%%%%%%%%%%%%%%%%%%%%%%%%%

%%%%%%%%%%%%%%%%%%%% mho 
\makecell{%
    \keys{W + tab  }} &

\makecell{%
    \keys{W + tab  }} &
$\mho$ (\tex{mho})
\\*
\midrule
%%%%%%%%%%%%%%%%%%%%%%%%%%%

%%%%%%%%%%%%%%%%%%%% wp 
\makecell{%
    \keys{P + tab  }} &

\makecell{%
    \keys{P + tab  }} &
$\wp$ (\tex{wp})
\\*
\midrule
%%%%%%%%%%%%%%%%%%%%%%%%%%%

%%%%%%%%%%%%%%%%%%%% infty
\makecell{%
    \keys{@ + @}} &

\makecell{%
    \keys{@ + @}} &
$\infty$ (\tex{infty} in \LaTeX\$)
\\*
\midrule
%%%%%%%%%%%%%%%%%%%%%%%%%%%

%%%%%%%%%%%%%%%%%%%% top        
\makecell{%
    \keys{T + tab }} &

\makecell{%
    \keys{T+ tab }} &
$\top$ (\tex{top })
\\*
\midrule
%%%%%%%%%%%%%%%%%%%%%%%%%%%

%%%%%%%%%%%%%%%%%%%% bot        
\makecell{%
    \keys{T+ tab }} &

\makecell{%
    \keys{T + tab }} &
$\bot$ (\tex{bot })
\\*
\midrule
%%%%%%%%%%%%%%%%%%%%%%%%%%%

%%%%%%%%%%%%%%%%%%%% clubsuit        
\makecell{%
    \keys{< + > + tab }} &

\makecell{%
    \keys{< + >  + tab }} &
$\clubsuit$ (\tex{clubsuit })
\\*
\midrule
%%%%%%%%%%%%%%%%%%%%%%%%%%%

%%%%%%%%%%%%%%%%%%%% diamondsuit         
\makecell{%
    \keys{< + > + tab }} &

\makecell{%
    \keys{< + >  + tab }} &
$\diamondsuit$ (\tex{diamondsuit  })
\\*
\midrule
%%%%%%%%%%%%%%%%%%%%%%%%%%%

%%%%%%%%%%%%%%%%%%%% heartsuit          
\makecell{%
    \keys{< + > + tab  }} &

\makecell{%
    \keys{< + >  + tab  }} &
$\heartsuit$ (\tex{heartsuit   })
\\*
\midrule
%%%%%%%%%%%%%%%%%%%%%%%%%%%

%%%%%%%%%%%%%%%%%%%% spadesuit          
\makecell{%
    \keys{< + > + tab   }} &

\makecell{%
    \keys{< + >  + tab   }} &
$\spadesuit$ (\tex{spadesuit   })
\\*
\midrule
%%%%%%%%%%%%%%%%%%%%%%%%%%%

%%%%%%%%%%%%%%%%%%%% flat           
\makecell{%
    \keys{b+ tab   }} &

\makecell{%
    \keys{b + tab  }} &
$\flat$ (\tex{flat  })
\\*
\midrule
%%%%%%%%%%%%%%%%%%%%%%%%%%%

%%%%%%%%%%%%%%%%%%%% natural           
\makecell{%
    \keys{\# + tab   }} &

\makecell{%
    \keys{\# + tab  }} &
$\natural$ (\tex{natural  })
\\*
\midrule
%%%%%%%%%%%%%%%%%%%%%%%%%%%

%%%%%%%%%%%%%%%%%%%% sharp            
\makecell{%
    \keys{\# + tab }} &

\makecell{%
    \keys{\# + tab }} &
$\sharp$ (\tex{sharp  })
\\*
\midrule
%%%%%%%%%%%%%%%%%%%%%%%%%%%

%%%%%%%%%%%%%%%%%%%% triangleq
\makecell{%
    \keys{@ + = + tab}} &

\makecell{%
    \keys{@ + = + tab}} &
$\triangleq$ (\tex{triangleq})
\\*
\midrule
%%%%%%%%%%%%%%%%%%%%%%%%%%%

%%%%%%%%%%%%%%%%%%%% dagger
\makecell{%
    \keys{{+} + tab }} &

\makecell{%
    \keys{{+} + tab }} &
$\dagger$ (\tex{dagger})
\\*
\midrule
%%%%%%%%%%%%%%%%%%%%%%%%%%%


    \makecell{\textbf{Variable sized operators}} \\
\midrule

%%%%%%%%%%%%%%%%%%%% int
\makecell{%
  \keys{I + tab}
} &

\makecell{%
  \keys{I + tab}
} &
$\int$ (\tex{int})
\\*
\midrule
%%%%%%%%%%%%%%%%%%%%%%%%%%%

%%%%%%%%%%%%%%%%%%%% iint
\makecell{%
  \keys{I + I +tab}
} &

\makecell{%
  \keys{I + I +tab}
} &
$\iint$ (\tex{iint})
\\*
\midrule
%%%%%%%%%%%%%%%%%%%%%%%%%%%

%%%%%%%%%%%%%%%%%%%% iiint
\makecell{%
  \keys{I + I + I + tab}
} &

\makecell{%
  \keys{I + I + I +tab}
} &
$\iiint$ (\tex{iiint})
\\*
\midrule
%%%%%%%%%%%%%%%%%%%%%%%%%%%

%%%%%%%%%%%%%%%%%%%% oint
\makecell{%
  \keys{@ + I}
} &

\makecell{%
  \keys{@ + I}
} &
$\oint$ (\tex{oint})
\\*
\midrule
%%%%%%%%%%%%%%%%%%%%%%%%%%%

%%%%%%%%%%%%%%%%%%%%%%%%%%% bigcup 
\makecell{%
  \keys{U + tab}} &

\makecell{%
  \keys{U + tab}} &
$\bigcup$ (\tex{bigcup})
\\*
\midrule
%%%%%%%%%%%%%%%%%%%%%%%%%%%

%%%%%%%%%%%%%%%%%%%%%%%%%%% bigcap 
\makecell{%
  \keys{N + tab}} &

\makecell{%
  \keys{N + tab}} &
$\bigcap$ (\tex{bigcap})
\\*
\midrule
%%%%%%%%%%%%%%%%%%%%%%%%%%%
    \makecell{\textbf{Arrow}} \\
\midrule

%%%%%%%%%%%%%%%%%%%% rightarrow
\makecell{%
\keys{- + >}} & 

\makecell{%
\keys{- + >}} & 
$\rightarrow$ (\tex{rightarrow})
\\*
\midrule
%%%%%%%%%%%%%%%%%%%%%%%%%%%

%%%%%%%%%%%%%%%%%%%% nrightarrow
\makecell{%
\keys{- + > + /}} & 

\makecell{%
\keys{- + > + /}} & 
$\nrightarrow$ (\tex{nrightarrow})
\\*
\midrule
%%%%%%%%%%%%%%%%%%%%%%%%%%%


%%%%%%%%%%%%%%%%%%%% longrightarrow
\makecell{%
\keys{- + - + >}} & 

\makecell{%
\keys{- + - + >}} & 
$\longrightarrow$ (\tex{longrightarrow})
\\*
\midrule
%%%%%%%%%%%%%%%%%%%%%%%%%%%

%%%%%%%%%%%%%%%%%%%% Rightarrow
\makecell{%
\keys{= + >}} & 

\makecell{%
\keys{= + >}} & 
$\Rightarrow$ (\tex{Rightarrow})
\\*
\midrule
%%%%%%%%%%%%%%%%%%%%%%%%%%%

%%%%%%%%%%%%%%%%%%%% nRightarrow
\makecell{%
\keys{= + > + /}} & 

\makecell{%
\keys{= + > + /}} & 
$\nRightarrow$ (\tex{nRightarrow})
\\*
\midrule
%%%%%%%%%%%%%%%%%%%%%%%%%%%


%%%%%%%%%%%%%%%%%%%% longRightarrow
\makecell{%
\keys{= + = + >}} & 

\makecell{%
\keys{= + = + >}} & 
$\Longrightarrow$ (\tex{Longrightarrow})
\\*
\midrule
%%%%%%%%%%%%%%%%%%%%%%%%%%%

%%%%%%%%%%%%%%%%%%%% leadsto
\makecell{%
\keys{\~{} + >}} & 

\makecell{%
\keys{\~{} + >}} & 
$\leadsto$ (\tex{leadsto})
\\*
\midrule
%%%%%%%%%%%%%%%%%%%%%%%%%%%

%%%%%%%%%%%%%%%%%%%% mapsto
\makecell{%
\keys{| + - + >}} & 

\makecell{%
\keys{| + - + >}} & 
$\mapsto$ (\tex{mapsto})
\\*
\midrule
%%%%%%%%%%%%%%%%%%%%%%%%%%%

%%%%%%%%%%%%%%%%%%%% longmapsto
\makecell{%
\keys{| + - + - + >}} & 

\makecell{%
\keys{| + - + - + >}} & 
$\longmapsto$ (\tex{longmapsto})
\\*
\midrule
%%%%%%%%%%%%%%%%%%%%%%%%%%%

%%%%%%%%%%%%%%%%%%%% leftarrow
\makecell{%
\keys{< + -}} & 

\makecell{%
\keys{< + -}} & 
$\leftarrow$ (\tex{leftarrow})
\\*
\midrule
%%%%%%%%%%%%%%%%%%%%%%%%%%%

%%%%%%%%%%%%%%%%%%%% leftrightarrow
\makecell{%
\keys{< + - + >}} & 

\makecell{%
\keys{< + - + >}} & 
$\leftrightarrow$ (\tex{leftrightarrow})
\\*
\midrule
%%%%%%%%%%%%%%%%%%%%%%%%%%%

%%%%%%%%%%%%%%%%%%%% uparrow
\makecell{%
\keys{< + - + Tab}} & 

\makecell{%
\keys{< + - + Tab}} & 
$\downarrow$ (\tex{uparrow})
\\*
\midrule
%%%%%%%%%%%%%%%%%%%%%%%%%%%

%%%%%%%%%%%%%%%%%%%% downarrow
\makecell{%
\keys{< + - + Tab + Tab}} & 

\makecell{%
\keys{< + - + Tab + Tab}} & 
$\downarrow$ (\tex{downarrow})
\\*
\midrule
%%%%%%%%%%%%%%%%%%%%%%%%%%%

%%%%%%%%%%%%%%%%%%%% updownarrow
\makecell{%
\keys{< + - + > + Tab}} & 

\makecell{%
\keys{< + - + > + Tab}} & 
$\updownarrow$ (\tex{updownarrow})
\\*
\midrule
%%%%%%%%%%%%%%%%%%%%%%%%%%%
    \makecell{\textbf{Fences}} \\
\midrule

%%%%%%%%%%%%%%%%%%%% langle rangle
\makecell{%
    \keys{< + \shift + Tab}} &

\makecell{%
    \keys{< + \shift + Tab}} &
$\langle\,\rangle$ (\tex{langle}\tex{rangle})
\\*
\midrule
%%%%%%%%%%%%%%%%%%%%%%%%%%%

%%%%%%%%%%%%%%%%%%%% lfloor rfloor
\makecell{%
    \keys{| + .}} &

\makecell{%
    \keys{| + .}} &
$\lfloor\, \rfloor$ (\tex{lfloor} \tex{rfloor})
\\*
\midrule
%%%%%%%%%%%%%%%%%%%%%%%%%%%

%%%%%%%%%%%%%%%%%%%% lceil rceil
\makecell{%
    \keys{| + '}} &

\makecell{%
    \keys{| + '}} &
$\lceil\, \rceil$ (\tex{lceil} \tex{rceil})
\\*
\midrule
%%%%%%%%%%%%%%%%%%%%%%%%%%%

%%%%%%%%%%%%%%%%%%%% norm
\makecell{%
    \keys{| + |}} &

\makecell{%
    \keys{| + |}} &
$\| \, \|$ (\tex{|} $\,$\tex{|})
\\*
\midrule
%%%%%%%%%%%%%%%%%%%%%%%%%%%
\end{xltabular}


\end{document}